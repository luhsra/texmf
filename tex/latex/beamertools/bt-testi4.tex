\documentclass{beamer}

\ifdefined\woschtheme%
  % beamer class setup (wosch style)
  \usecolortheme{seahorse}
  \usecolortheme{rose}
  \useoutertheme{infolines}
\else
  % beamer class setup (i4 style)
  \usecolortheme{rose}
  \usetheme{i4}
\fi

\setbeamertemplate{navigation symbols}{}

% we use UTF8
\usepackage[utf8]{inputenc}

% font setup
\usepackage[T1]{fontenc}
\usepackage{lmodern}
\renewcommand{\sfdefault}{cmbr} 
\usepackage[scaled=0.85]{beramono}

% Misc
\usepackage{xspace}

% load the beamertools package
\usepackage[everything,physicalpagesinpdftoc]{beamertools}
\usetikzlibrary{positioning}
\def\beamertools{\texttt{beamertools}\xspace}
\def\key#1{{\color{black!50}\texttt{/bt/#1}}}
\def\tikzkey#1{{\color{black!50}\texttt{/tikz/#1}}}

% typesetting LaTeX code together with result
\usepackage{showexpl}
\lstloadlanguages{[LaTeX]TeX}
\lstdefinestyle{expl}{
  basicstyle=\scriptsize\ttfamily,
  language=[LaTeX]Tex,
  breaklines=true,
  breakautoindent=true,
  breakindent=2ex,
  moredelim=**[is][\bfseries]{''}{''},
  moredelim=[is][\color{beamergreen}\slshape]{'}{'},
  numbers=none, numbersep=2ex,xleftmargin=0pt, 
  morekeywords={structure,alert,tikz,node,lstset,tikzset,path,draw},
  morekeywords=[2]{ bi,ei,ii,bii,eii,biii,eiii,iiad,iida,Alert,ALERT,Structure,STRUCTURE,sample,Sample,SAMPLE,
                    btBlock,btUseExtraItemSep,btAddExtraItemSep,btset,btConvertTo,btInsertFileModDate,
                    btLstInputEmph, btVFill, btHL, btPrevFrameTitle, btPrevFrameSubtitle, btPrevShortFrameTitle},
  keywordstyle=[2]{\color{i4red}},
  morekeywords=[3]{onslide, {scale content}, {add font}},
  keywordstyle=[3]{\color{i4red}},
}

% set expl-specific listings options
\lstset{rframe={},pos=b,explpreset={style=expl}}

% set standard listing options
\lstset{
  style=expl,
}

\lstMakeShortInline{!}


\begin{document}

\title[beamertools]{Documentation of the \beamertools package}
\author[dl] {Daniel Lohmann}
\date{\btInsertFileModDate{\jobname.tex}}

\begin{frame}[plain]
  \titlepage
\end{frame}

\section{Introduction}

\begin{frame}{Purpose of the package}
  \btUseExtraItemSep[\medskipamount]
  \bi
    \ii The \beamertools package provides a convenient interface to certain extensions and patches I have developed for my \texttt{beamer} presentations, especially the lecture slides for (G)SPiC and BS 
    \ii I created this package after figuring out, that
      \bi
        \ii my \texttt{preamble.tex} files become way too long, way too redundant and way too complicated
        \ii I found certain repeating code patterns in my lecture presentations that could be shortened quite a bit by better abstractions
        \ii I always wanted to write an own \LaTeX{} package :-)
      \ei
  \ei
\end{frame}
\begin{frame}[fragile,t]{Package loading and options}
  \bi
    \ii Package options are processed with !pgfkeys!
      \bi
        \ii Example: !\usepackage[autonotes,notikz]{beamertools}! 
        \ii<2> All \lstinline{pgfkeys} features (e.g., styles) can be employed:
           !\usepackage[spic]{beamertools}! 
      \ei
      \ii The following \alt<1>{options}{styles} are available (sorry, no real docu yet):
  \ei
  \only<1>{\lstinputlisting[linerange=31-38]{beamertools.sty}}
  \only<2>{\lstinputlisting[linerange=48-54]{beamertools.sty}}
\end{frame}

\section{Shortcut and Helper Macros}

\begin{frame}[fragile]{Shortcuts for List Environments}
Shortcuts for the !itemize! environment: !\bi! ... !\ii! ... !\ei!
\begin{LTXexample}[pos=r]
  \bi
    \ii<+-> Level 1
      \bi
        \ii Level 2
      \ei
    \ii<+-> Level 1 again
  \ei
\end{LTXexample}
Variants to skip one or two levels (for compact lists):
\begin{LTXexample}[pos=r]
  \bii
    \ii This is a level 2 item
    \ii This is a level 2 item
  \eii
  \biii
    \ii This is a level 3 item
    \ii This is a level 3 item
  \eiii
\end{LTXexample}
Variants for advantage/disadvantage lists (easy to redefine):
\begin{LTXexample}[pos=r]
  \bii
    \iiad This is an advantage
    \iida This is a disadvantage
  \eii
\end{LTXexample}
\end{frame}

\begin{frame}[fragile,allowframebreaks]{Spacing in List Environments}
Better spacing between items, weighted by the itemize level.
\smallskip
\bi
  \ii The !\btAddExtraItemSep['<sep>=\smallskipamount']! command advances !\itemsep! by !'<sep> * (3 - itemize level)'!. 
\ei
\begin{LTXexample}[pos=r]
\bii
  \ii Normal Spacing
  \ii Normal Spacing
  \btAddExtraItemSep
  \ii Extended Spacing
    \bi
      \ii Normal Spacing
      \ii Normal Spacing
    \ei
  \ii Extended Spacing
\eii
\end{LTXexample}
\bi
  \ii It has to be applied inside the !itemize! environment and only affects the current level.
\ei
\framebreak
\bi
  \ii The !\btUseExtraItemSep['<sep>=\smallskipamount']! command patches the !itemize! environment, so that !\btAddExtraItemSep['<sep>']! is invoked implicitly: 
\ei
\begin{LTXexample}[pos=r]
\btUseExtraItemSep[1ex]
\bi
  \ii Extended Spacing
  \ii Extended Spacing
    \bi
      \ii Extended Spacing
      \ii Extended Spacing
    \ei
  \ii Extended Spacing
\ei
\end{LTXexample}
\bi
  \ii If applied at the begin of a !frame! environement, it affects all lists on the frame. 
  \ii This can be great to fine-tune the spacing.
\ei
\end{frame}

\begin{frame}[fragile]{Additional text styles}
  \framesubtitle{Very useful}
Some additional variants of the \lstinline{\alert}, \lstinline{\structure}, and a (all new) \lstinline{\sample} command. All accept an \lstinline{<'overlay spec'>}:
\begin{LTXexample}[pos=r]
  \bii
    \ii This is \alert{text}
    \ii This is \Alert{text}
    \ii This is \ALERT{text}
  \eii
\end{LTXexample}
\begin{LTXexample}[pos=r]
  \bii
    \ii This is \structure{text}
    \ii This is \Structure{text}
    \ii This is \STRUCTURE{text}
  \eii
\end{LTXexample}
\begin{LTXexample}[pos=r]
  \bii
    \ii This is \sample{text}
    \ii This is \Sample{text}
    \ii This is \SAMPLE{text}
  \eii
\end{LTXexample}
\end{frame}

\begin{frame}[fragile]{Previous frame title}
  The macros !\btPrevFrameTitle!, !\btPrevFrameSubtitle!, !\btPrevShortFrametitle! provide the title, subtitle and short title of the previous frame (look back to see what was the title):
\begin{LTXexample}[pos=r]
  \bii
    \ii \btPrevFrameTitle 
    \ii \btPrevFrameSubtitle
    \ii \btPrevShortFrameTitle
  \eii
\end{LTXexample}
\end{frame}

\section{Block Environments}
\begin{frame}[fragile, allowframebreaks]{The \texttt{btBlock} Environment}
  \btUseExtraItemSep[\medskipamount]
  \bi
    \ii General structure
      \begin{lstlisting}[autogobble]
        \begin{btBlock}<'overlay spec'>['pgfkeys key=val list']{'title'}
          'block content'
        \end{btBlock}
      \end{lstlisting}
        \ii Minimal Example

\begin{LTXexample}[pos=r]
\begin{btBlock}[]{Block}
  Something important
\end{btBlock}
\end{LTXexample}

        \ii Using block types: \key{type=alert|example|normal}

\begin{LTXexample}[pos=r]
\begin{btBlock}[type=alert]{Block}
  Something important
\end{btBlock}
\end{LTXexample}

\begin{LTXexample}[pos=r]
\begin{btBlock}[type=example]{Block}
  Something important
\end{btBlock}
\end{LTXexample}

        \ii Scaling: \key{scale content=} and \key{scale=}
          \bi
            \ii \key{scale content=} keeps width, but scales block content so that more stuff fits into it \par
\begin{LTXexample}[pos=r]
\begin{btBlock}[scale content=0.7]{Block}
  more info
\end{btBlock}
\end{LTXexample}

            \ii \key{scale=} scales block "as is", so that block consumes less space \par
\begin{LTXexample}[pos=r]
\begin{btBlock}[scale=0.7]{Block}
  more info
\end{btBlock}
\end{LTXexample}

          \ei
        \ii Setting block width: \key{text width=}\par
\begin{LTXexample}[pos=b]
\begin{btBlock}[text width=5cm]{Block}
  more info
\end{btBlock}
\end{LTXexample}
\hrule
\par
\begin{LTXexample}[pos=b]
\begin{btBlock}[text width=0.8\textwidth]{Block}
  more info
\end{btBlock}
\end{LTXexample}

        \ii Horizontal alignment: \key{align=left|right|center}\par
\begin{LTXexample}[pos=b]
\begin{btBlock}[text width=0.8\textwidth,align=right]{Block}
  more info
\end{btBlock}
\end{LTXexample}
\hrule
\par
\begin{LTXexample}[pos=b]
\begin{btBlock}[scale=0.8, align=center]{Block}
  more info
\end{btBlock}
\end{LTXexample}

        \ii Beamer-Block options: \key{rounded} and \key{shadow}\par
\begin{LTXexample}[pos=r]
\begin{btBlock}[shadow=false]{Block}
  more info
\end{btBlock}
\end{LTXexample}
\hrule
\par
\begin{LTXexample}[pos=r]
\begin{btBlock}[rounded=false]{Block}
  more info
\end{btBlock}
\end{LTXexample}
        \ii Setting defaults: The \key{every block} style 

\begin{LTXexample}[pos=r]
\btset{every block/.style={
  rounded, shadow=false, 
  scale=0.8, center}
}
\begin{btBlock}{Block}
  more info
\end{btBlock}

\bigskip

\btset{every block/.append style={shadow, alert}}
\begin{btBlock}{Block}
  more info
\end{btBlock}
\end{LTXexample}

  \ei
\end{frame}

\begin{frame}[fragile]{Wosch-compatible blocks}
\bi
\ii
  If you load the package with the \key{woschblocks} option, 
  the following environments will be defined on the base of \texttt{btBlock}.\newline
  {\footnotesize (Note that \texttt{btBlock} options can still be specified)}
\par
\begin{LTXexample}[pos=r]
\begin{bearblock}{Block}
  more info
\end{bearblock}
\medskip
\begin{ovalblock}{Block}
  more info
\end{ovalblock}
\medskip
\def\shadow{true}
\begin{codeblock}[scale content=0.8]{Block}
  more info
\end{codeblock}
\end{LTXexample}
\ii
  These should be fully compatible to the ones Wosch uses in his slides (including handling of the \texttt{\textbackslash shadow} macro)

\ei

\end{frame}

\section{TikZ Stuff}
\begin{frame}[fragile]{Additional styles for TikZ}
\bi
\ii Add to current font (instead of replacing it) \tikzkey{add font=\emph{font command}}
\ii Scale inner content of a node \tikzkey{scale content=\emph{factor}}\par
\ii Use beamer overlays with TikZ styles \tikzkey{onslide=}
\par
\begin{LTXexample}
\tikz\node[%
    font=\ttfamily,
    onslide=<1>{draw=blue}, 
    onslide=<2->{fill=red!50, add font=\bfseries},
    onslide=<3>{scale content=1.5}
  ]{Attention!};
\end{LTXexample}
\ei
\end{frame}

\begin{frame}[fragile]{Piecewise appearing for TikZ}
\bi
\ii Use beamer overlays for visibility \tikzkey{visible on=}
\par
\begin{LTXexample}
\begin{tikzpicture}[every node/.style={fill=i4red!30, draw=i4red}]
  \node{Foo}
    child[visible on=<2->]{node {Bar}}
    child[visible on=<3->]{node {Baz}}
  ;  
\end{tikzpicture}
\end{LTXexample}
\ii<4-> Default implementation is based on \\
  \tikzkey{opacity=0}:
\begin{lstlisting}
\tikzset{
  invisible/.style={opacity=0},
  visible on/.style={alt=#1{}{invisible}},
}
\end{lstlisting}
\ei
\PutAt<4->{(7.5cm,4cm)}{%
  \begin{btBlock}[text width=5cm, scale content=0.7]{Advantage: Elements are always there}
    \bi
      \ii Image size does not depend on the overlay step
      \ii Named nodes are always defined (for coordinate calculation)
   \ei
  \end{btBlock}
}
\end{frame}

\begin{frame}[fragile]{Piecewise appearing for TikZ (cont.) }
\bi
\ii By overriding the \tikzkey{invisible} style, the "invisible" appearance can be customized (e.g., to dim elements instead) 
\par
\begin{LTXexample}
\tikzset{invisible/.style={opacity=0.2}}
\begin{tikzpicture}[every node/.style={fill=i4red!30, draw=i4red}]
  \node{Foo}
    child[visible on=<2->]{node {Bar}}
    child[visible on=<3->]{node {Baz}}
  ;  
\end{tikzpicture}
\end{LTXexample}
\ei
\end{frame}


\section{Extensions to \texttt{listings}}

\begin{frame}[fragile]{Highlighting lines in Listings}
\begin{LTXexample}[pos=r]
\lstset{language=C, numbers=left}
\begin{lstlisting}[
  autogobble,
  linebackgroundcolor={%
    \btLstHL{4}%
    \btLstHL<1>{1-2,5-6}%
    \btLstHL<2>{7}%
  }]
    /**
    * Prints Hello World. äöü
    **/
    #include <stdio.h>

    int main(void) {
       printf("Hello World!");  
       return 0;
    }
\end{lstlisting}
\end{LTXexample}


\end{frame}

\begin{frame}[fragile]{Highlighting lines in listings from external files}
\begin{LTXexample}[pos=r]
\btLstInputEmph[language=C, numbers=left]{3,6-7}{hello.c}
\end{LTXexample}

\end{frame}

\begin{frame}[fragile]{Highlighting single elements in listings}
\scriptsize !\btHL<'overlay spec'>['tikz key=val list']! highlights till the end of a group (no line breaks, though).  Hence, it can be as a ordinary font command with listings:
\begin{LTXexample}[pos=b]
\bii
  \ii Some {text mit \btHL highlighting}, overlays are {\btHL<2>[red!20]also} possible.
\eii
\lstset{language=C, autogobble}
\begin{lstlisting}[
    moredelim={**[is][\btHL<1->]{@1}{@}},
    moredelim={**[is][{\btHL<2>}]{@2}{@}}
  ]
    #include @2<stdio.h>@

    int @1main@(void) {
       printf("Hello World!");  
       return 0;
    }
\end{lstlisting}
\end{LTXexample}
\end{frame}

\begin{frame}[fragile]{Highlighting single elements in listings}
\bii
\ii !\btHL<'overlay spec'>['tikz key=val list']! actually draws the content inside a TikZ node, so you can play with named nodes and other options:
\eii
\begin{LTXexample}[pos=b]
\begin{lstlisting}[language=C, autogobble, numbers=left,
    moredelim={**[is][{%
      \btHL[name=X, remember picture, onslide=<2->{fill=red!50}]%
    }]{@}{@}},
  ]
    @int main (void)@ {
       printf("Hello World!");  
       return 0;
    }
\end{lstlisting}
% main() is typset into the node (X):
\tikz[remember picture, overlay]{
  \path<2> node[red, above right=3mm of X](L){This is the entry point};
  \draw<2>[->, red, shorten >=5pt] (L.west)--(X);
}
\end{LTXexample}
\end{frame}

\section{Misc Macros and Stuff}

\begin{frame}[fragile]{Miscellaneous}
\bi
\ii Dimension conversions with \lstinline|\btConvertTo{'dim'}{'dim value'}|:
\par
\begin{LTXexample}[pos=r]
100pt=\btConvertTo{mm}{100pt}mm
\end{LTXexample}
\ii Get file modification date of some file (ISO format) with \lstinline|\btInsertFileModDate{'file'}|:
\par
\begin{LTXexample}[pos=r]
This document was changed on 
\btInsertFileModDate{\jobname.tex}
\end{LTXexample}
\ii Real vertical fill to bottom with \lstinline|\btVFill|, stackable
\begin{LTXexample}[pos=r]
\btVFill
\fbox{Always at Bottom}
\end{LTXexample}
\btVFill
\fbox{Always at Bottom}
\ei
  
\end{frame}


\end{document}

